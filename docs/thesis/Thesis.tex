% MSc Thesis of Giovanni Demasi - 987062
% Politecnico di Milano (PoliMi) - School of Industrial and Information Engineering

\documentclass{Configuration_Files/PoliMi3i_thesis}

%------------------------------------------------------------------------------
%	REQUIRED PACKAGES AND  CONFIGURATIONS
%------------------------------------------------------------------------------

% CONFIGURATIONS
\usepackage{parskip} % For paragraph layout
\usepackage{setspace} % For using single or double spacing
\usepackage{emptypage} % To insert empty pages
\usepackage{multicol} % To write in multiple columns (executive summary)
\setlength\columnsep{15pt} % Column separation in executive summary
\setlength\parindent{0pt} % Indentation
\raggedbottom

\newcommand{\tabitem}{~~\llap{-}~~}

% PACKAGES FOR TITLES
\usepackage{titlesec}
% \titlespacing{\section}{left spacing}{before spacing}{after spacing}
\titlespacing{\section}{0pt}{3.3ex}{2ex}
\titlespacing{\subsection}{0pt}{3.3ex}{1.65ex}
\titlespacing{\subsubsection}{0pt}{3.3ex}{1ex}
\usepackage{color}

% PACKAGES FOR LANGUAGE AND FONT
\usepackage[english]{babel} % The document is in English  
\usepackage[utf8]{inputenc} % UTF8 encoding
\usepackage[T1]{fontenc} % Font encoding
\usepackage[11pt]{moresize} % Big fonts

% PACKAGES FOR IMAGES
\usepackage{graphicx}
\usepackage{transparent} % Enables transparent images
\usepackage{eso-pic} % For the background picture on the title page
\usepackage{subfig} % Numbered and caption subfigures using \subfloat.
\usepackage{tikz} % A package for high-quality hand-made figures.
\usetikzlibrary{}
\graphicspath{{./Images/}} % Directory of the images
\usepackage{caption} % Coloured captions
\usepackage{xcolor} % Coloured captions
\usepackage{amsthm,thmtools,xcolor} % Coloured "Theorem"
\usepackage{float}

% STANDARD MATH PACKAGES
\usepackage{amsmath}
\usepackage{amsthm}
\usepackage{amssymb}
\usepackage{amsfonts}
\usepackage{bm}
\usepackage[overload]{empheq} % For braced-style systems of equations.
\usepackage{fix-cm} % To override original LaTeX restrictions on sizes

% PACKAGES FOR TABLES
\usepackage{tabularx}
\usepackage{longtable} % Tables that can span several pages
\usepackage{colortbl}

% PACKAGES FOR ALGORITHMS (PSEUDO-CODE)
\usepackage{algorithm}
\usepackage{algorithmic}
\usepackage{listings}
\renewcommand\lstlistingname{Listing}
\renewcommand\lstlistlistingname{List of Listings}

% PACKAGES FOR REFERENCES & BIBLIOGRAPHY
\usepackage[colorlinks=true,linkcolor=black,anchorcolor=black,citecolor=black,filecolor=black,menucolor=black,runcolor=black,urlcolor=black]{hyperref} % Adds clickable links at references
\usepackage{cleveref}
\usepackage[square, numbers, sort&compress]{natbib} % Square brackets, citing references with numbers, citations sorted by appearance in the text and compressed
\bibliographystyle{abbrvnat} % I may use a different style adapted to your field

% OTHER PACKAGES
\usepackage{pdfpages} % To include a pdf file
\usepackage{afterpage}
\usepackage{lipsum} % DUMMY PACKAGE
\usepackage{fancyhdr} % For the headers
\fancyhf{}

% Input of configuration file.
% Set the geometric layout of the document
\usepackage{geometry}
\geometry{
  top=3cm,
  left = 2.0cm,
  right = 2.0cm,
  bottom=2cm,
  headheight= 2cm,
  headsep= 0cm,
}
\raggedbottom 


% Custom theorem environments
\declaretheoremstyle[
  headfont=\color{black}\normalfont\bfseries,
  bodyfont=\color{black}\normalfont\itshape,
]{colored}

\captionsetup[figure]{labelfont={color=black}} % Set colour of the captions
\captionsetup[table]{labelfont={color=black}} % Set colour of the captions
\captionsetup[algorithm]{labelfont={color=black}} % Set colour of the captions

\theoremstyle{colored}
\newtheorem{theorem}{Theorem}[section]
\newtheorem{proposition}{Proposition}[section]

% Enhances the features of the standard "table" and "tabular" environments.
\newcommand\T{\rule{0pt}{2.6ex}}
\newcommand\B{\rule[-1.2ex]{0pt}{0pt}}

% Algorithm description
\newcounter{algsubstate}
\renewcommand{\thealgsubstate}{\alph{algsubstate}}
\newenvironment{algsubstates}{
    \setcounter{algsubstate}{0}%
    \renewcommand{\STATE}{%
    \stepcounter{algsubstate}%
    \Statex {\small\thealgsubstate:}\space}
    }{}
    
% Custom theorem environment
\newcolumntype{L}[1]{>{\raggedright\let\newline\\\arraybackslash\hspace{0pt}}m{#1}}
\newcolumntype{C}[1]{>{\centering\let\newline\\\arraybackslash\hspace{0pt}}m{#1}}
\newcolumntype{R}[1]{>{\raggedleft\let\newline\\\arraybackslash\hspace{0pt}}m{#1}}

% Custom itemize environment
\setlist[itemize,1]{label=$\bullet$}
\setlist[itemize,2]{label=$\circ$}
\setlist[itemize,3]{label=$-$}
\setlist{nosep}

% Set separation of columns 
\setlength{\columnsep}{30pt}

% Create command for background pic
\newcommand\BackgroundPic{% Adding background picture
	\put(230,358){
		\parbox[b][\paperheight]{\paperwidth}{%
			\vfill
			\centering
			\transparent{0.4}
			\vfill
}}}

% Set indentation
\setlength\parindent{0pt}

% Custom title commands
\titleformat{\section}
{\color{black}\normalfont\Large\bfseries}
{\color{black}\thesection.}{1em}{}
\titlespacing*{\section}
{0pt}{2ex}{1ex}

\titleformat{\subsection}
{\color{black}\normalfont\large\bfseries}
{\color{black}\thesubsection.}{1em}{}
\titlespacing*{\subsection}
{0pt}{2ex}{1ex}

% Custom headers and footers
\pagestyle{fancy}
\fancyhf{}
      
\fancyfoot{}
\fancyfoot[C]{\thepage} % page
\renewcommand{\headrulewidth}{0mm} % headrule width
\renewcommand{\footrulewidth}{0mm} % footrule width

\makeatletter
\patchcmd{\headrule}{\hrule}{\color{black}\hrule}{}{} % headrule
\patchcmd{\footrule}{\hrule}{\color{black}\hrule}{}{} % footrule
\makeatother

% -> Create the header
\chead[C]{
\centering
\textbf{ Executive summary} \hfill \textbf{\author}
\centerline{\rule{1.0\textwidth}{0.4pt}}
}

%----------------------------------------------------------------------------
%	NEW COMMANDS DEFINED
%----------------------------------------------------------------------------

\definecolor{codegreen}{rgb}{0,0.6,0}
\definecolor{codegray}{rgb}{0.5,0.5,0.5}
\definecolor{codepurple}{rgb}{0.58,0,0.82}
\definecolor{backcolour}{rgb}{0.95,0.95,0.92}

\lstdefinestyle{mystyle}{
    %backgroundcolor=\color{backcolour},
    breakatwhitespace=false,
    breaklines=true,
    captionpos=b,
    keepspaces=true,
    %numbers=left,
    numbersep=5pt,
    showspaces=false,
    showstringspaces=false,
    showtabs=false,
    tabsize=2,
    %xleftmargin=5.0ex
}

\lstset{style=mystyle}

%----------------------------------------------------------------------------
%	ADDITIONAL PACKAGES
%----------------------------------------------------------------------------

%----------------------------------------------------------------------------
%	ADDITIONAL DEFINITIONS AND COMMANDS
%----------------------------------------------------------------------------

%----------------------------------------------------------------------------
%	DOCUMENT
%----------------------------------------------------------------------------

\begin{document}
    \nocite{*}

    \fancypagestyle{plain}{%
        \fancyhf{} % Clear all header and footer fields
        \fancyhead[RO,RE]{\thepage} %RO=right odd, RE=right even
        \renewcommand{\headrulewidth}{0pt}
        \renewcommand{\footrulewidth}{0pt}}

%----------------------------------------------------------------------------
%	TITLE PAGE
%----------------------------------------------------------------------------

    \pagestyle{empty} % No page numbers
    \frontmatter % Use roman page numbering style (i, ii, iii, iv...) for the preamble pages

    \puttitle{
        title=An FPGA Toolchain for \\ Graph Neural Network \\ Acceleration using \\ High-Level Synthesis, % Title of the thesis
        name=Giovanni Demasi, % Author Name and Surname
        course=Computer Science and Engineering, % Study Programme
        ID  = 987062,  % Student ID number (numero di matricola)
        advisor= Prof. Fabrizio Ferrandi, % Supervisor name
        coadvisor={Serena Curzel, Michele Fiorito}, % Co-Supervisor name
        academicyear={2022-23},  % Academic Year
    } % These info will be put into your Title page

%----------------------------------------------------------------------------
%	PREAMBLE PAGES: ABSTRACT (inglese e italiano), EXECUTIVE SUMMARY
%----------------------------------------------------------------------------
    \startpreamble
    \setcounter{page}{1} % Set page counter to 1

% ABSTRACT IN ENGLISH
    \chapter*{Abstract}
    Here goes the Abstract in English of your thesis followed by a list of keywords.
    The Abstract is a concise summary of the content of the thesis (single page of text)
    and a guide to the most important contributions included in your thesis.
    The Abstract is the very last thing you write.
    It should be a self-contained text and should be clear to someone who hasn't (yet) read the whole manuscript.
    The Abstract should contain the answers to the main scientific questions that have been addressed in your thesis.
    It needs to summarize the adopted motivations and the adopted methodological approach as well as the findings of your work and their relevance and impact.
    The Abstract is the part appearing in the record of your thesis inside POLITesi,
    the Digital Archive of PhD and Master Theses (Laurea Magistrale) of Politecnico di Milano.
    The Abstract will be followed by a list of four to six keywords.
    Keywords are a tool to help indexers and search engines to find relevant documents.
    To be relevant and effective, keywords must be chosen carefully.
    They should represent the content of your work and be specific to your field or sub-field.
    Keywords may be a single word or two to four words.
    \\
    \\
    \textbf{Keywords:} here, the keywords, of your thesis % Keywords

% ABSTRACT IN ITALIAN
    \chapter*{Abstract in Lingua Italiana}
    Qui va l'Abstract in lingua italiana della tesi seguito dalla lista di parole chiave.
    \\
    \\
    \textbf{Parole chiave:} qui, vanno, le parole chiave, della tesi % Keywords (italian)

%----------------------------------------------------------------------------
%	LIST OF CONTENTS/FIGURES/TABLES/SYMBOLS
%----------------------------------------------------------------------------

% TABLE OF CONTENTS
    \thispagestyle{empty}
    \tableofcontents % Table of contents
    \thispagestyle{empty}
    \cleardoublepage

%-------------------------------------------------------------------------
%	THESIS MAIN TEXT
%-------------------------------------------------------------------------
% In the main text of this thesis it is possible to write the chapters in two different ways:
%
%(1) It is possible to write:
%    \chapter{Title of the chapter}
%    *body of the chapter*
%
%(2) It is also possible to write the chapter in a separated .tex file and then include it in the main file with the following command:
%    \chapter{Title of the chapter}
%    \input{chapter_file.tex}
%
% Especially for long thesis, the second option is the recommended one.

    \addtocontents{toc}{\vspace{2em}} % Add a gap in the Contents, for aesthetics
    \mainmatter % Begin numeric (1,2,3...) page numbering

% ##########################################################################
% CHAPTER ONE - INTRODUCTION
% ##########################################################################

    \chapter{Introduction}
    \label{ch:chapter_one}%
    Over the past few years, deep learning has significantly revolutionized various machine learning tasks,
spanning from image classification and video processing to speech recognition and natural language understanding.
Traditionally, these tasks have predominantly operated within the Euclidean space, where data is typically
represented.
For instance, in image analysis applications, images can be considered as functions defined on the Euclidean space (plane) and sampled on a grid.
Nevertheless, a growing number of applications now generate data from non-Euclidean domains~\cite{DBLP:journals/corr/BronsteinBLSV16},
presenting it in the form of complex graphs with intricate relationships and interdependencies among objects.
The inherent complexity of graph data has posed considerable challenges for existing machine learning algorithms.
Consequently, there has been a surge of studies focusing on extending deep learning techniques to accommodate
and leverage graph data.

Graph neural networks (GNNs) have been introduced in response to the growing demand for learning tasks involving
graph data, which encompasses extensive relational information among its elements.
These neural models effectively capture the interdependence among graph nodes by employing message passing mechanisms.

Optimizing and accelerating the capabilities of Graph Neural Networks is necessary due to their increasingly popularity, particularly in domains characterized by vast amounts of data,
such as social networks and chemistry.
In particular, inference in GNNs refers to the time the model takes to make predictions after training.
The duration of the inference process determines the speed at which queries are answered, and researchers strive to minimize this time span.

In applications of deep learning that prioritize low latency, Field-programmable Gate Arrays (FPGAs) outperform other computing devices, such as CPUs and GPUs.
FPGAs offer the advantage of being fine-tuned to the application to strike the optimal balance between power efficiency and meeting performance requirements.

Due to this reason, researchers have been actively pursuing the development of new FPGA accelerators for Graph Neural Networks (GNNs) in recent times.

The conventional approach to hardware design involves a combination of manual coding and automated processing.
In particular, first the functional units are implemented in a programming language such as C/C++, then they are transformed into a Hardware Description Language (HDL) using commercial High-Level Synthesis (HLS) tools.
Following functional verification, the HDL kernels are forwarded to downstream logic synthesis and physical design tools, and finally integrated into a system.
However, this method demands significant effort and relies heavily on the expertise of the designers, leading to varying quality of results.

To address the challenge of accelerating GNNs on FPGAs without having extensive knowledge in hardware design, the objective of this thesis is to develop a comprehensive toolchain that, starting from PyTorch~\cite{DBLP:journals/corr/abs-1912-01703},
a cutting-edge high-level programming framework for creating neural network algorithms based on the Python programming language, enables the
automatic generation of a Graph Neural Networks (GNNs) FPGA accelerator with minimal effort required.

The suggested toolchain represents an enhancement of the SODA toolchain~\cite{9786533}.
It operates by transforming the PyTorch model, provided as input, into a multi-level intermediate representation
(MLIR)~\cite{9370308} utilizing Torch-MLIR~\cite{torch_mlir}, an MLIR based compiler toolkit for PyTorch programs.
This MLIR representation is then passed to the SODA framework to conduct hardware/software partitioning of the algorithm
specifications and architecture-independent optimizations.
Following this, the framework generates a low-level IR (LLVM IR) specifically tailored for the hardware generation engine,
PandA-Bambu~\cite{9586110}.

In pursuit of the thesis goal, various optimizations were adopted throughout the process.
Specifically, efforts were made to optimize specific computations in Graph Neural Networks.
As these networks often deal with massive graph sizes, the computation time and memory requirements are substantial.
Consequently, a significant portion of the research focuses on optimizing the computation phase of Graph Neural Networks using
custom optimizations.

This analysis aims to provide valuable insights for future research endeavors, enabling the development of solutions
to overcome these limitations and further enhance the proposed toolchain.

TODO: add something about results

%While the intended purpose of the toolchain is to be general, the experimental phase primarily focused on two specific
%types of Graph Neural Networks: Graph Isomorphism Networks (GIN)~\cite{xu2019powerful} and Graph Convolutional Networks (GCN)~\cite{DBLP:journals/corr/KipfW16}.
%These models were sourced from reliable GitHub implementations and were modified as necessary.
%
%The GCN model~\cite{pygcn}, designed for node classification task and written in pure PyTorch, held particular importance for the
%experimental phase as it served as the basis for the resulting accelerator.
%On the other hand, the GIN model~\cite{ogb_gnn_models}, designed for graph classification task and written in PyTorch Geometric~\cite{DBLP:journals/corr/abs-1903-02428},
%a library built upon PyTorch for easier development and training of Graph Neural Networks, did not progress through
%the final step of the proposed toolchain.
%This was due to some incompatibilities between PyTorch Geometric and Torch-MLIR, which are integral parts of this thesis research.

\section{Contributions}
\label{sec:contributions}%


\section{Thesis structure}
\label{sec:thesis_structure}%

Chapter~\ref{ch:chapter_one} introduced the context of the thesis, its objective, and its goals.
Chapter~\ref{ch:chapter_two} presents background about Graph Neural Networks, how they work, an explanation of the GNN types used in the thesis, and the type of tasks that they can perform, including some of their applications.
Additionally, it presents the SODA framework, the starting point for this thesis's proposed toolchain.
Chapter~\ref{ch:chapter_three} contains an overview of related work; other Graph Neural Network acceleration frameworks are analyzed, underlying their differences compared to the proposed approach and their limitations.
Chapter~\ref{ch:chapter_four} formulates the problem statement, summarizes the open issues of the research objective, and explains the expected impact.
Chapter~\ref{ch:chapter_five} explains how the problem has been faced and what technologies have been used.
It contains a detailed description of the proposed toolchain and its working method.
Chapter~\ref{ch:chapter_six} lists all the performed experiments, gives the necessary information to reproduce them and contains their outcomes and the issues and limitations encountered.
Finally, Chapter~\ref{ch:conclusions} presents overall considerations of the study, both with the main achievements obtained and the most notable obstacles faced.
Along with this, potential improvements for future studies are considered.

% ##########################################################################
% CHAPTER TWO - BACKGROUND
% ##########################################################################


    \chapter{Background}
    \label{ch:chapter_two}%
    % The \label{...}% enables to remove the small indentation that is generated, always leave the % symbol.
    This chapter provides essential background to understand of the thesis content and objectives.
It begins by introducing the graph data structure, which is crucial for comprehending Graph Neural Networks.
Additionally, the chapter provides an introduction to Graph Neural Networks, outlining their capabilities and exploring various applications.
Furthermore, it introduces two essential tools, SODA and Bambu, which are integral parts of the SODA Toolchain that served as the foundation for this research.

\section{Graphs}
\label{sec:graphs}%

\textit{Graphs} are data structures representing a collection of objects, known as vertices or nodes, and a set of edges connecting them~\cite{DBLP:journals/corr/abs-1812-08434}.
In a graph, the edges can be either directed or undirected, as shown in Figure~\ref{fig:directed_vs_undirected}, and they typically connect two vertices, which may or may not be distinct.
The vertices represent entities or elements, and the edges represent their relationships or connections.

\begin{figure}[b]
    \centering
    \subfloat[Directed Graph\label{fig:directed_graph}]{
        \includegraphics[width=0.2\textwidth]{Images/directed_graph}
    }
    %\quad
    \hspace{0.15\textwidth}
    \subfloat[Undirected Graph\label{fig:undirected_graph}]{
        \captionsetup{width=.4\textwidth}
        \includegraphics[width=0.2\textwidth]{Images/undirected_graph}
    }
    %\caption[Shorter caption]{This is a very long caption you don't want to appear in the List of Figures.}
    \caption{Example of directed and undirected graphs}
    \label{fig:directed_vs_undirected}
\end{figure}

Graphs serve as a versatile tool for describing diverse forms of data.
For example, molecules, the fundamental units of matter, are composed of atoms and electrons arranged in three-dimensional space.
In this intricate structure, all particles interact with each other.
However, when a pair of atoms are stably positioned at a specific distance, we refer to their connection as a covalent bond.
These bonds with distinct atomic distances can vary in nature, such as single or double bonds.
Representing this complex three-dimensional object as a graph offers a practical and widely adopted abstraction, where atoms are nodes and covalent bonds act as edges~\cite{DBLP:journals/corr/DuvenaudMAGHAA15}.

Social networks provide another domain where graphs are used: in fact, they serve as valuable tools for examining patterns within the collective behavior of people, institutions, and organizations.
By representing individuals as nodes and their relationships as edges, we can construct a graph that effectively captures groups of people and their interconnectedness.

\subsection{Graph Representation}
\label{subsec:graph_representation}

Here I am going to talk about how graphs can be represented.
Especially the ways encountered during the research: adjacency matrix, COO and CSR\@.

%TODO: add an image with a graph and its representation in the 3 different format

\section{Graph Neural Networks}
\label{sec:graph_neural_networks}%

%TODO: add an example image of a GNN

Graph neural networks (GNNs) are deep learning techniques that operate on graph-structured data.
Thanks to their impressive performance, GNNs have recently gained significant popularity as a widely adopted method for graph analysis~\cite{KERAMATFAR2022100401}.
Figure~\ref{fig:google_scholar} illustrates the steady growth in the number of publications related to Graph Neural Networks (GNNs) on Google Scholar from 2015 to 2022.
The data were collected by querying papers containing the specific words "Graph Neural Network" in their whole content and aggregating them on a yearly basis.
The increasing trend reflects the rising interest and research activity in the field of GNNs over the years.

\begin{figure}[t]
    \centering
    \includegraphics[width=0.7\textwidth]{Images/google_scholar}
    %\caption[Shorter caption]{This is a very long caption you don't want to appear in the List of Figures.}
    \caption{Number of GNN publications on Google Scholar per year}
    \label{fig:google_scholar}
\end{figure}

Graph Neural Networks (GNNs) are designed to process graph data and consist of multiple interconnected layers.
At its core, a GNN is an algorithm that exploits the connectivity within a graph to understand and represent the relationships between nodes.
By relying on the graph's structure, the GNN iteratively processes input edge, vertex, and graph feature vectors, which encode known attributes and transforms them into output feature vectors that capture the desired predictions.
Each Graph Neural Network typically encompasses three main stages: pre-processing, iterative updates and decoding or readout~\cite{DBLP:journals/corr/abs-2010-00130}.
\begin{enumerate}
    \item \textbf{Pre-processing}: this initial step, while optional, involves transforming the input feature vectors and graph structure representation through a pre-processing procedure.
    \item \textbf{Iterative updates}: following pre-processing, the feature vectors of each edge and vertex undergo iterative updates using aggregate-combine functions.
          For edge updates, attributes from the edge itself, connected vertices, and the graph are aggregated and combined to generate a new edge feature vector.
          Similarly, vertex updates involve aggregating feature vectors from neighboring vertices $\mathcal{N}(v)$ and combining them to obtain a new feature vector.
          This iterative process gradually incorporates relationships between increasingly distant nodes and edges, allowing for multi-hop updates.
          Furthermore, the graph may coarsen through pooling~\cite{DBLP:journals/corr/abs-1806-08804} (i.e. selective reduction or adjustment of either the graph structure or the neighborhood set of each node) in each subsequent layer, or the neighborhood set may change via layer sampling~\cite{DBLP:journals/corr/HamiltonYL17} (i.e. coarsening the graph from one layer to the next, leading to a reduction in the number of nodes that need to be processed during aggregation and combination steps).
    \item \textbf{Decoding or readout}: once the graph possesses a global feature vector, it is updated once upon completion of edge and node updates.
          The final output can be an edge/node embedding, representing specific information about each edge or node in a low-dimensional feature vector format, or a graph embedding that summarizes the entire output graph.
\end{enumerate}
Performing these stages on large and sparse graphs can introduce dynamic computational data flow and numerous irregular memory access patterns.

GNNs, as previously said, are structured into layers, each representing an iteration in the update process described earlier.
This layering allows information to propagate across nodes, enabling the influence of distant nodes.
Consequently, the appropriate number of layers in a GNN will vary depending on the significance of relationships among distant nodes in a specific application.
The commonly adopted range for the number of GNN layers is 1 to 5, as an excessive number of layers can introduce undesired problems such as feature over-smoothing, vanishing gradients, or over-fitting~\cite{DBLP:journals/corr/abs-1801-07606}.

Graph Neural Networks are a group of neural networks which are designed to solve different tasks.
Prediction tasks on graphs can generally be classified into three categories: graph-level, node-level, and edge-level predictions~\cite{sanchez-lengeling2021a}.

In a graph-level task, the objective is to predict the property or characteristic of an entire graph.
For instance, when considering a molecule represented as a graph, attributes might be aimed to be predicted such as its likelihood of binding to a receptor associated with a specific disease.
This assignment is comparable to image classification tasks, where the objective is to assign a label to an entire image.
Similarly, in text analysis, sentiment analysis serves as a similar problem where the goal is to determine a complete sentence's overall mood or emotion in one go.

Node-level tasks involve predicting the identity or function of individual nodes within a graph.
One example of a node-level task is node classification in a social network.
Given a social network graph where nodes represent individuals and edges represent relationships between them, the task is to predict the demographic attributes or characteristics (e.g., age, gender, occupation) of each node based on their connection patterns and features.
Drawing an analogy to image processing, node-level prediction problems can be compared to image segmentation tasks, where the objective is to assign labels to each pixel in an image based on its role.
Similarly, in text analysis, a comparable task would involve predicting the parts of speech for each word in a sentence, such as identifying whether a word is a noun, verb, adverb, and so on.


The remaining prediction task in graphs pertains to edge prediction.
One example of an edge-level task is link prediction in a social network.
Given a graph representing a social network where, as before, in node-level tasks, nodes correspond to individuals and edges represent relationships between them, the edge-level task aims to predict missing or potential connections between nodes.
This can involve predicting the likelihood of a future friendship or the probability of a collaboration between individuals based on their shared characteristics or mutual connections in the network.

Different popular Graph Neural Network architectures have been proposed recently, some of which are more suitable for some tasks than others.
A summary of two types of GNNs is provided in the following subsections.

\subsection{Graph Convolutional Network}
\label{subsec:graph_convolutional_network}%

A graph convolutional network (GCN)~\cite{DBLP:journals/corr/KipfW16, daigavane2021understanding} is a type of neural network architecture explicitly designed to operate on graph-structured data.
GCNs aim to learn node representations by aggregating and combining information from neighboring nodes in the graph.
The core idea behind GCNs is to perform convolution-like operations on the graph, where the convolutional filters are defined based on the graph's adjacency matrix or other graph-specific structures.
This enables GCNs to capture and leverage the structural information encoded in the graph to make predictions or perform downstream tasks.
GCNs have demonstrated effectiveness in various applications, including node classification, link prediction, and graph classification.

Given an undirected graph $\mathcal{G} = (V, E)$, where $V$ represents the set of nodes (vertices), and $E$ represents the set of edges, with an adjacency matrix $\tilde{A}=A+I_N$, where $I_N$ is the identity matrix, the layer-wise propagation rule in a GCN can be expressed as:
\begin{equation}
    \label{eq:gcn_convolution}
    H^{(l+1)} = f \left( \tilde{D}^{-\tfrac{1}{2}}  \tilde{A}  \tilde{D}^{-\tfrac{1}{2}}  H^{(l)}  W^{(l)} \right)
\end{equation}

Where $H^{(l)} \in \mathbb{R}^{N \times D}$ is the input node features matrix, $W^{(l)}$ is a layer-specific learnable weight matrix, $\tilde{D}$ is the degree matrix defined as $\tilde{D}_{ii} = \sum_{j} \tilde{A}_{ij}$, and $f(\cdot)$ represents a non-linear activation function applied element-wise, such as $ReLU(\cdot) = max(0, \cdot)$.
The equation above demonstrates the propagation of node features through graph convolution, where the adjacency matrix $\tilde{A}$ captures the connectivity information of the graph, $\tilde{D}^{-\tfrac{1}{2}}$ normalizes the adjacency matrix, and $H^{(l)}  W^{(l)}$ performs a linear transformation of node features.
The resulting $H^{(l+1)}$ represents the updated node representations after the graph convolution operation.
In practice, multiple graph convolutional layers can be stacked to capture increasingly complex relationships and further refine the node representations.

\subsection{Graph Isomorphism Network}
\label{subsec:graph_isomorphism_network}%

A Graph Isomorphism Network (GIN)~\cite{xu2019powerful, daigavane2021understanding} is a type of neural network architecture designed to operate on graph-structured data by capturing graph isomorphism, which is the property of two graphs having the same structure, inspired by the Weisfeiler-Lehman (WL) graph isomorphism test~\cite{xu2019powerful}.
GINs aim to learn node representations that are invariant under graph isomorphism, enabling them to generalize across different graphs with similar structures.

The learned vertex features from GIN-Conv can be directly utilized for tasks such as node classification and link prediction.
The model is based on the following rule:
\begin{equation}
    \label{eq:gin_function}
    h_v^{(k+1)} = MLP^{(k)} \left( \left( 1 + \epsilon^{(k)} \right) \cdot h_v^{(k)} + \sum_{u \in \mathcal{N}(v)} h_u^{(k)} \right)
\end{equation}

Where $h_v^{(k)}$ represents the initial node representation of node $v$, $\mathcal{N}(v)$ represents the neighborhood of node $v$, $\epsilon$ is a learnable
parameter or a fixed scalar, $MLP( \cdot )$ represents a Multi Layer Perceptron and $h_v^{(k+1)}$ represents the updated node representations.

In the neighborhood aggregation process of GINs, each node's representation is updated by considering its own representation and its neighbors' representations.
The neighborhood aggregation is performed through the MLP operation, followed by non-linear activation.

GINs are trained using graph-level objectives, such as graph classification or property prediction, and aim to learn invariant representations under graph isomorphism, allowing them to generalize well to unseen graphs with similar structures.
However, even if the node embeddings acquired through GIN can be directly applied to tasks such as node classification and link prediction, in the case of graph classification tasks, it is necessary to use a Readout function that takes individual node embeddings as input and produces the embedding representation for the entire graph.

The Readout function is then utilized to generate the overall representation of the graph, leveraging the individual vertex representations.
By concatenating the results from all iterations of GINConv, the final graph representation is obtained as:
\begin{equation}
    \label{eq:gin_readout}
    h_G = CONCAT \left( READOUT \left( \left\{ h_v^{(k)} | v \in G \right\} \right) | k = 0, 1, ..., K \right)
\end{equation}

Where $READOUT$ in~\ref{eq:gin_function} can be replaced with a sum operator in order to generalize the WL test~\cite{xu2019powerful}.


\section{SODA Toolchain}
\label{sec:soda}%

SODA~\cite{9786533} is a software-defined accelerator synthesizer.
It enables the creation of highly specialized accelerators from algorithms designed in high-level programming frameworks.
The synthesizer comprises a compiler-based frontend that interfaces with high-level programming frameworks, applying advanced optimizations.
It also includes a compiler-based backend responsible for generating Verilog code and interfacing with external tools to compile the final design, which can be applied to application-specific integrated circuits (ASICs) or field-programmable gate arrays (FPGAs).

SODA's exceptional power lies in its ability to offer a fully automated end-to-end hardware compiler, eliminating the need for human intervention and any modifications to the input code.
This framework seamlessly integrates with high-level Python frameworks by accepting their input descriptions, which are then translated by the frontend into a high-level intermediate representation (IR).
Leveraging the multi-level intermediate representation (MLIR), the frontend facilitates hardware/software partitioning of algorithm specifications and performs architecture-independent optimizations.
Following this, it generates a low-level IR (LLVM IR) that is utilized by the hardware generation engine, PandA-Bambu~\cite{9586110}.
PandA-Bambu can accept LLVM IR as input, making it a cutting-edge open-source HLS tool.
Throughout the entire SODA toolchain, compiler passes are employed to implement optimizations at all levels, greatly influencing the generated hardware designs' performance, area, and power characteristics.

\subsection{SODA-OPT Frontend}
\label{subsec:soda_frontend}%

SODA-OPT, the high-level compiler frontend of the SODA synthesizer, performs search, outlining, optimization, dispatching, and acceleration passes on the input program.
Its primary objective is to prepare the program for hardware synthesis, targeting either FPGAs or ASICs.
To accomplish these tasks, SODA-OPT relies on and extends the MLIR framework.
MLIR is a framework that facilitates the development of reusable, extensible, and modular compiler infrastructure by defining dialects.
These dialects serve as self-contained intermediate representations (IRs) that adhere to the meta-IR syntax of MLIR.
By utilizing dialects, code can be modeled at different levels of abstraction, allowing for specialized representations that aid in specific compiler optimizations.

 Code regions selected for hardware acceleration undergo an optimization pipeline that progressively lowers them through various MLIR dialects until they are ultimately translated into an LLVM IR format tailored explicitly for hardware synthesis.
On the other hand, the host module is lowered into an LLVM IR file containing runtime calls to control the generated custom accelerators.

\subsection{SODA Synthesizer Backend}
\label{subsec:soda_backend}%

Bambu, the SODA synthesizer backend, harnesses cutting-edge HLS techniques to produce accelerator designs using the low-level LLVM IR generated by the SODA frontend.
Bambu supports multiple frontends based on standard compilers such as GCC or CLANG.
It constructs an internal IR to execute HLS steps and generates designs in HDL formats, such as Verilog or VHDL.
In addition to synthesizable HDL, Bambu can automatically generate testbenches for verification purposes.
Using Bambu, the SODA synthesizer can target both FPGAs and ASICs.

Bambu is optimized to handle a broad range of C and C++ constructs while also being able to process LLVM IR through its internal Clang frontend.
Through SODA-OPT, Bambu can be connected with MLIR code.
The LLVM IR generated after SODA-OPT's high-level optimizations undergoes explicit restructuring for HLS, resulting in more efficient accelerators than direct translation from MLIR to LLVM IR.

\section{Conclusion}
\label{sec:background_conclusion}%
This chapter has presented the foundational concepts necessary for understanding the subsequent contents of this thesis.
It provided a concise overview of the broad domain of Graphs and Graph Neural Networks, explicitly focusing on the architectures of Graph Convolutional Networks and Graph Isomorphism Networks.
Additionally, the chapter introduced SODA and PandA-Bambu, which will be further investigated within the context of the proposed design flow for the creation of GNNs FPGA-based accelerators.

The following chapter is dedicated to an analysis of scientific literature on hardware acceleration for Graph Neural Networks.
This analysis primarily focuses on publications concerning FPGA-based implementations and design flows that leverage High-Level Synthesis techniques.


% ##########################################################################
% CHAPTER THREE - RELATED WORK
% ##########################################################################


    \chapter{Related Work}
    \label{ch:chapter_three}%
    Accelerating Graph Neural Networks (GNNs) has become a subject of intense interest within the research community, encompassing the exploration of ASIC and FPGA accelerators.
In this chapter, a comprehensive examination is conducted on cutting-edge Graph Neural Networks FPGA accelerators and design flows based on High-Level Synthesis (HLS).
As explained in Chapter 6, particular emphasis has been placed on optimizing matrix-matrix multiplication during this thesis research study.
Consequently, this chapter also delves into the relevant literature concerning various approaches to Matmul optimization.

\section{Chapter structure}
\label{sec:related_work_structure}
This chapter contains several sections.
Firstly, it presents the software frameworks utilized to accelerate Graph Neural Network computations.
The following section provides an overview of state-of-the-art hardware accelerators, categorized based on their architecture types~\cite{DBLP:journals/corr/abs-2010-00130}.

Subsequently, a section summarizes an accelerator implemented using High-Level Synthesis (HLS). This accelerator is separated from the hardware accelerators as it adopts HLS as the design flow proposed in this thesis.

Additionally, this chapter includes a summary of a solution that aimed to accelerate GNN using both software and hardware approaches.

As mentioned earlier, optimizing the matrix-matrix multiplication operation was a significant aspect of this research.
Thus, a dedicated section focuses on state-of-the-art optimizations for matrix-matrix multiplication, especially those related to technologies similar to the ones employed in this thesis.

Finally, the chapter concludes with a comprehensive summary of the cutting-edge accelerators presented.

\section{Software accelerators}
\label{sec:related_work_software_accelerators}

The challenges posed by GNN processing have led to inefficiencies in traditional deep neural network (DNN) libraries and graph processing frameworks.
This is primarily due to the alternating computational phases characteristic of GNNs.
While DNN libraries excel in accelerating combination operations within vertices and edges, they need help with aggregation tasks.
On the other hand, graph processing libraries effectively handle irregular memory accesses during graph traversal but assume simplistic operations at the vertices, which is not the case in GNNs. Recent research studies tried to bridge the gap by adapting the DNN libraries to overcome Graph Neural Network challenges.

The two main software frameworks trying to accelerate Graph Neural Networks computation are PyTorch Geometric~\cite{DBLP:journals/corr/abs-1903-02428} and Deep Graph Library~\cite{DBLP:journals/corr/abs-1909-01315}.
They both provide a lot of examples and code for multiple GNN architectures providing optimizations that could work for the acceleration of both training and inference.

PyTorch Geometric is a PyTorch-based library specifically designed for deep learning on input data with irregular structures, including graphs, point clouds, and manifolds.
In addition to offering comprehensive graph data structures and processing techniques, it incorporates many state-of-the-art methods from relational learning and 3D data processing domains.
PyTorch Geometric achieves remarkable data throughput by introducing efficient handling of mini-batches containing input examples of varying sizes and efficiently handling sparsity through specialized GPU scatter and gather kernels, which operate on all edges and nodes concurrently, as opposed to relying on sparse matrix multiplication kernels.
A key aspect of PyG involves defining a message-passing interface encompassing message and update functions for neighborhood aggregation and combination and multiple pooling operations.

DGL is a recently developed library that seamlessly integrates with TensorFlow, PyTorch, or MXNet.
It introduces three essential functions: message for aggregating edges, update and reduce for aggregating and combining at the nodes.
DGL adopts a matrix multiplication approach to enhance performance and harnesses specialized kernels designed for GPUs or TPUs.
Specifically, both sampled dense-dense and sparse matrix multiplications and options for node, edge, or feature parallelization are considered.
DGL intelligently selects the optimal parallelization scheme using heuristics, considering various factors, including the input graph.
It distills the computational patterns of GNNs into a set of generalized sparse tensor operations, which facilitate extensive parallelization.
By prioritizing the graph as the central programming abstraction, DGL enables transparent optimizations.
Furthermore, through a framework-neutral design philosophy, DGL allows users to effortlessly port and leverage existing components across multiple deep learning frameworks.

The approach used by DGL outperformed PyTorch Geometric in training Graph Neural Networks, as stated in their paper~\cite{DBLP:journals/corr/abs-1909-01315}.
However, both libraries target CPU and GPU architectures.
Knowing the extreme computational power of FPGA, the field of hardware accelerators started gaining more and more interest, with the expectation of having GNN hardware accelerators capable of outperforming the performance of CPU-GPU targeting libraries.

\section{Hardware accelerators}
\label{sec:hardware_accelerators}

As discussed in Section~\ref{sec:related_work_software_accelerators}, software accelerators optimize the execution of GNNs in CPU-GPU platforms, commonly found in various computing systems, leading to substantial speed improvements in inference and training processes.

However, the research field has raised questions about the feasibility of custom hardware accelerators in overcoming the challenges of GNN computing and achieving order-of-magnitude enhancements.
Consequently, numerous hardware accelerators with different architecture types have emerged, aiming to address the intensive computational demands and alternating patterns required by GNNs.

\subsection{Unified architecture accelerators}\label{subsec:unified-architecture-accelerators}

A unified architecture refers to a design approach where the FPGA fabric is configured to be versatile and flexible, allowing it to handle various applications and tasks.
Instead of having specialized and fixed hardware modules for specific functions, a unified architecture enables the FPGA to reconfigure its resources to dynamically adapt to different computation requirements.

\cite{DBLP:journals/corr/abs-1908-10834} presents Autotuning-Workload-Balancing GCN (AWB-GCN) to accelerate GCN inference.
This accelerator endorses a proactive adaptation to the structural sparsity inherent in GNNs. The authors support their design by analyzing the power-law distribution found in most graphs, positing that certain parts of the computation will exhibit density. In contrast, others will be extraordinarily sparse, leading to imbalances.

In order to tackle this problem, the architecture devises a custom matrix multiplication engine that efficiently supports skipping zeros.
In particular, three hardware-based autotuning techniques to address the imbalance have been suggested: dynamic distribution smoothing, remote switching, and row remapping.

Specifically, AWB-GCN continuously monitors the sparse graph pattern, dynamically adjusts the workload distribution among a substantial number of processing elements, and reuses the optimal configuration upon convergence.
Data from memory is directed through a task distributor and queue (TDQ) to a collection of processing elements (PEs) and accumulators.
The TDQ has two designs tailored for scenarios with moderate or high sparsity.
Given AWB-GCN's emphasis on GCNs featuring linear aggregation functions, the authors suggest prioritizing combination processing, as this typically reduces the number of features and subsequently minimizes the operations performed during aggregation.
Additionally, AWB-GCN incorporates a fine-grained pipelining mechanism to effectively overlap the execution of combination and aggregation, even within the same layer.

However, at the heart of the AWB-GCN architecture lies the management of load balancing at three levels of granularity: distribution smoothing to handle local utilization fluctuations among PEs, remote switching for minor crests, and row remapping for prominent crests.
At the beginning of the processing, rows are evenly distributed among processing elements.
Throughout each round of calculation, distribution smoothing equalizes the workloads among neighboring PEs.
The architecture of AWB-GCN effectively monitors the runtime PE utilization by tracking the number of pending tasks in task queues.
It continually offloads the work from more burdened PEs to their less occupied neighbors, up to 3-hop neighbors.

Remote switching is implemented to tackle regional clustering, wherein the process facilitates partial or complete workload exchanges between underutilized and overloaded PEs.
An auto-tuner dynamically determines the switch fraction at runtime, relying on the PE utilization observed in each round.
The accelerator retains the switch strategies employed in the current round and iteratively optimizes them based on utilization information gathered in the subsequent round.
As a result, after several rounds of auto-tuning, the switch strategy that best aligns with the sparse matrix structure is attained and is then utilized for the remaining rounds, leading to nearly perfect PE utilization.

Lastly, the evil-row remapping technique redistributes the evil row to the most under-loaded PEs in troughs, allowing the neighboring PEs to assist.
Row remapping is initiated based on demand after each round.
The auto-tuner assesses the utilization gaps between the most overloaded and under-loaded PEs and decides if their gaps exceed remote switching capability.
If so, row remapping is executed as a solution.




% ##########################################################################
% CHAPTER FOUR - PROBLEM FORMULATION
% ##########################################################################


    \chapter{Problem Formulation}
    \label{ch:chapter_four}%
    This chapter aims to formulate the problem rigorously, explaining the thesis's objective, motivation, and research questions that guided this work.

\section{Graph Neural Network acceleration}
\label{sec:gnn_acceleration}%

Graph neural network acceleration refers to designing and implementing hardware accelerators and co-processors to speed up the training or inference of GNNs.
A GNN accelerator aims at optimizing the execution of GNN computations, which involve iterative message-passing between nodes in a graph to update their representations based on neighboring nodes' features.
Hardware acceleration aims to improve the performance, efficiency, and capabilities of computing systems by offloading specific tasks or computations to specialized hardware components.

In particular, this thesis focuses on improving GNNs inference time, by designing specialized hardware to efficiently perform the computation-intensive operations involved in GNNs, such as matrix multiplications, aggregations, and non-linear activation functions.

A small example explains the potential impact of this thesis's objective.
Let us consider a recommendation system, in which the goal is to predict what items a user might be interested in based on their past interactions and preferences.
An example is Netflix suggesting what to watch next based on previously watched movies and ratings.
This problem can be represented as a graph, where users and movies (items) are nodes and interactions between users and movies are edges.
A GNN is an optimal choice for modeling recommendation systems, as it can effectively capture relationships and interactions between users and items.
However, as the number of users and items increases, the computational complexity of GNNs can become a significant bottleneck.
One way to address the bottleneck issue is by using a specialized GNN accelerator, which is purpose-built to handle GNN computations on extensive graphs efficiently.
By exploiting parallelism and data locality in GNN operations, it enables faster and more energy-efficient processing of the graph.

Utilizing the GNN accelerator, the recommendation system can offer real-time recommendations to users, even on edge devices with limited computational capabilities.
Ultimately, the GNN accelerator enhances the efficiency and scalability of the recommendation system, resulting in quick and precise recommendations to users. It also reduces computational and memory overhead.

\section{Motivation and objective}
\label{sec:motivation}%

As mentioned in Chapter~\ref{ch:chapter_three}, different state-of-the-art accelerators exist and use different approaches to improve GNNs' performance.
Even if various alternatives are available, almost all GNN acceleration research has implicitly focused on either developing highly efficient schemes tailored for specific GNN models or aiming for generality and flexibility to accommodate various types of GNNs with less efficiency.

The primary challenge driving the research in this thesis lies in creating a framework that optimizes performance and efficiency while retaining the necessary flexibility to adapt to diverse graph sizes, characteristics, and GNN algorithms.

The main research questions that led to this thesis can be summarised as follow:
\begin{enumerate}
    \item How is it possible to design hardware accelerators that exploit the unique characteristics of GNN computation?
    \item How is it possible to synthesize accelerators, starting from high-level programming languages, without being a hardware design expert?
    \item What are the bottlenecks of GNNs that can slow down their inference time?
    \item What are the most effective low-level optimizations for GNNs that can improve their efficiency without sacrificing model accuracy?
    %\item How does the proposed GNN hardware accelerator compare to existing software solutions?
    \item Can an automated design be generalized to different GNN models and datasets while ensuring high performance?
\end{enumerate}


Consequently, the main objectives of the thesis are:
\begin{enumerate}
    \item Investigate existing GNN models and identify bottlenecks that hinder their inference efficiency.
    \item Develop an FPGA toolchain for GNN acceleration, allowing seamless integration with various GNN models and datasets.
    \item Explore low-level optimizations for GNNs to improve hardware performance.
    \item Synthesize hardware accelerators tailored to GNNs, leveraging parallelism and memory optimizations to accelerate graph computations.
    \item Implement and evaluate the proposed GNN accelerators on FPGA\@.
\end{enumerate}




% ##########################################################################
% CHAPTER FIVE - TOOLCHAIN
% ##########################################################################


    \chapter{FPGA Toolchain for Graph Neural Network Acceleration}
    \label{ch:chapter_five}%
    The main contribution of this thesis is represented by the design of a toolchain for Graph Neural Network acceleration on FPGA leveraging High-Level Synthesis.

This chapter explains in detail how the toolchain has been designed and how it can be used to build GNN accelerators to enhance inference performance.

The core component of the toolchain is the synthesizer, enclosing SODA-OPT~\cite{9786533} and PandA-Bambu~\cite{9586110}, upon which the research team responsible for supervising this thesis has dedicated significant efforts over the past years.
The primary objective of this thesis is to enhance SODA-OPT and Bambu to bridge GNN models written in high-level frameworks, such as PyTorch, to FPGA architectures.

Figure~\ref{fig:toolchain} illustrates the entire design flow, with the steps involved in the GNN acceleration process.
In particular, firstly, the GNN model is implemented in PyTorch, one of the most popular and powerful frameworks for Neural Network implementations.
Subsequently, the model is passed as input to Torch-MLIR, a crucial middle step that enables the generation of the MLIR representation.
This intermediate representation serves as input for the synthesizer, where, once the frontend optimization is complete, the refined version proceeds to the backend, where the actual GNN accelerator is effectively produced, ready to enhance inference performance on FPGA architectures.

The following Sections provide a comprehensive and in-depth exploration of each step within the proposed design flow.
This detailed breakdown highlights the various possibilities inherent in the toolchain and outlines the recommended procedures necessary to achieve the optimal outcome for GNN acceleration.
The thesis aims to equip researchers and practitioners in Graph Neural Networks with the necessary insights and understanding to harness the full potential of this toolchain and unleash the power of FPGA acceleration.

In conclusion, this thesis represents a significant advancement in Graph Neural Network acceleration.
By designing a refined toolchain and bridging the gap between high-level frameworks and FPGA architectures, this research contributes to the broader field of artificial intelligence.
It reinforces the potential of FPGA-based accelerators in revolutionizing the inference performance of GNN models.
The implications of this work offer a solid foundation for further exploration and advancements in the field of hardware acceleration for deep learning applications.

\begin{figure}[t]
    \centering
    \includegraphics[height=0.6\textwidth]{Images/toolchain}
    \caption{FPGA Toolchain for Graph Neural Network Acceleration}
    \label{fig:toolchain}
\end{figure}

\section{PyTorch}
\label{sec:toolchain-pytorch}%

PyTorch~\cite{DBLP:journals/corr/abs-1912-01703} is an open-source deep learning framework widely used for building and training artificial neural networks for various machine learning tasks.

The first step of the toolchain is to design and implement the Graph Neural Network model in PyTorch.
Doing so involves defining the GNN model architecture and writing the necessary forward pass to compute node and graph-level representations.
Once having defined the model, the next step is training the GNN using standard PyTorch techniques, such as defining a loss function, setting up an optimizer, and performing backpropagation to optimize the model parameters.

\subsection{GNN models}
\label{subsec:gnn_models}%

Two main models have been used for this thesis: the Graph Isomorphism Network from OGB~\cite{NEURIPS2020_fb60d411, ogb_gnn_models}, written using PyTorch Geometric~\cite{DBLP:journals/corr/abs-1903-02428}, and the Graph Convolutional Network~\cite{DBLP:journals/corr/KipfW16, pygcn}, written using PyTorch~\cite{DBLP:journals/corr/abs-1912-01703}.

Most research and experiments have been conducted using the GCN model.
The GCN class, reported below, is characterized by two Graph Convolutional layers.

\begin{lstlisting}[language=Python,label={lst:gcn-class}, numbers=left, xleftmargin=1em, caption=Class of GCN model]
import torch.nn as nn
import torch.nn.functional as F
from pygcn.layers import GraphConvolution

class GCN(nn.Module):
    def __init__(self, nfeat, nhid, nclass, dropout):
        super(GCN, self).__init__()

        self.gc1 = GraphConvolution(nfeat, nhid)
        self.gc1 = GraphConvolution(nhid, nclass)
        self.dropout = dropout

    def forward(self, x, adj):
        x = F.relu(self.gc1(x, adj))
        x = F.dropout(x, self.dropout, training=self.training)
        x = self.gc2(x, adj)
        return F.log_softmax(x, dim=1)
\end{lstlisting}

The forward function of each layer, reported below, is mainly characterized by two matrix multiplications.

\begin{lstlisting}[language=Python,label={lst:gcn-layer-forward}, numbers=left, xleftmargin=1em, caption=Forward function of GCN layer]
    def forward(self, input, adj):
        support = torch.mm(input, self.weight)
        output = torch.spmm(adj, support)
        if self.bias is not None:
            return output + self.bias
        else:
            return output
\end{lstlisting}

\subsection{Datasets}
\label{subsec:gnn_datasets}%

OGB provides different datasets that can be used with their models.
The one used for this thesis is called \textit{ogbg-molhiv}, a molecular property prediction dataset.
In each graph representing a molecule, nodes correspond to atoms, and edges represent chemical bonds.
The input node features consist of nine dimensions, encompassing information like atomic number, formal charge, and whether the atom is part of a ring.
The binary classification task consists in achieving precise predictions of target molecular properties, for example, determining whether a molecule inhibits HIV replication or not.

The dataset used for the GCN model is the \textit{Cora} one.
This dataset contains 2708 scientific publications, categorized into one of the seven classes considered.
The citation network contains 5429 links.
Each publication in the dataset is represented by a binary-valued word vector, indicating the absence or presence of the corresponding word from a dictionary of 1433 unique words.
The task is a multiclass classification, in which, given a paper, the objective is to classify it into one of the seven classes correctly.

\section{Torch-MLIR}
\label{sec:toolchain-torch_mlir}%

Torch-MLIR~\cite{torch_mlir} offers compiler support for transitioning from the PyTorch ecosystem to the MLIR ecosystem.

The steps Torch-MLIR follows to go from PyTorch to MLIR are shown in Figure~\ref{fig:torch-mlir}.
In particular, the flow followed in this thesis has been highlighted with blue arrows.
There are two starting points of the flow: TorchScript and LazyTensorCode.
The one used for this research, which is also the most tested one, is TorchScript.
TorchScript~\cite{torchscript} offers a way to generate serializable and optimizable models directly from PyTorch code.

The TorchScript representation is then converted to MLIR using the built-in conversion of Torch-MLIR. The result MLIR can use different dialects, but the one used for this thesis is the Linalg dialect, which serves as input for the next phase of the toolchain.

\begin{figure}[t]
    \centering
    \includegraphics[height=0.5\textwidth]{Images/torch-mlir}
    \caption{Torch-MLIR flow}
    \label{fig:torch-mlir}
\end{figure}

\subsection{From PyTorch to TorchScript}
\label{subsec:pytorch-to-torchscript}%

Since Torch-MLIR implicitly uses the TorchScript representation to go from PyTorch to MLIR, the first part of the research consisted of a deep analysis of the Graph Neural Network models to make them compatible with TorchScript.

This task required more effort for the GIN model due to its use of PyTorch Geometric than the GCN model, which only uses pure PyTorch.
In particular, the required adaptations, that have been used for both models, have been made general aiming to make them versatile for various applications.
They are documented below, with a small example for each of them:

\begin{itemize}
    \item[-] The GNN layer class, if created as a subclass of the Message Passing class, must be marked as Jittable whenever it is used.
\begin{lstlisting}[language=Python,label={lst:jittable}]
self.convs.append(GINConv(emb_dim).jittable())
\end{lstlisting}
    \item[-] The propagate function, if used, need its parameters to be explicitly annotated using one of the two available options: through the definition of a dictionary or through a comment.
\begin{lstlisting}[language=Python,label={lst:propagate-annotation}]
propagate_type = {'x': Tensor, 'edge_attr': Tensor}
\end{lstlisting}
    \item[-] It can happen that TorchScript is not able to recognize the correct type of variables.
    In this case, it is necessary to use an assertion to explicitly declare that the variable is an instance of the correct type.
\begin{lstlisting}[language=Python,label={lst:isinstance-assertion}]
assert isinstance(edge_embedding, Tensor)
\end{lstlisting}
    \item[-] A common approach to speed up the training and inference steps is to use batched data.
    Unfortunately, TorchScript does not support forward functions that take as input a batch.
    For this reason, the forward function must receive Tensors as input, thus the batch must be split into its component.
\begin{lstlisting}[language=Python,label={lst:splitted-forward}]
def forward(self, x, edge_index, edge_attr):
\end{lstlisting}
    \item[-] The parameters of the forward function must be explicitly annotated with their type.
    If not declared, it is assumed to be of type Tensor.
\begin{lstlisting}[language=Python,label={lst:forward-annotation}]
def forward(self, x: Tensor, edge_index: Tensor,
            edge_attr: Tensor) -> Tensor:
\end{lstlisting}
    \item[-] TorchScript always expects an integer literal for the index, this is because indexing is only supported with integer literals.
    For this reason, cycles that do not use integer literals must be changed into enumeration.
\begin{lstlisting}[language=Python,label={lst:enumeration}]
for idx, layer in enumerate(self.convs):
\end{lstlisting}
\end{itemize}

\subsection{Torch-MLIR Compilation}
\label{subsec:torch-mlir-compilation}%

Once having designed, implemented, made compatible with TorchScript, and trained the GNN model in PyTorch, it is possible to use the \lstinline{torch_mlir.compile} API to obtain the MLIR representation of the model.
In particular, this API takes three parameters as input: the GNN model, an input example of the model and the desired output type.
The Graph Neural Network model must have been already trained, being ready for inference.
The second parameter, the input example of the model, is an arbitrary input similar to the one that would be given for inference purposes.
It is required because, by default, the implicit Jit function called by Torch-MLIR to script the model and obtain a script module, involves compiling the forward method and recursively compiling any methods, submodules, and functions called within the forward method. This results in a JIT IR which is converted to the torch dialect which is almost in a 1:1 correspondence.
The torch dialect is then lowered into one of the three available output dialects: linalg, tosa, mhlo.
The purpose of the last parameter is to choose which of these three dialects has to be used for the output MLIR.

An additional parameter that can be used is related to the tracing.
There are two ways in which it is possible to obtain a TorchScript representation: \lstinline{torch.jit.script} and \lstinline{torch.jit.tracing}.
The compile API of Torch-MLIR uses the first one by default.
Instead, if the option use tracing is set to True, JIT tracing is used.
The behavior of the two functions is slightly different.
Tracing only captures functions and modules that lack data dependencies and untracked external dependencies.
It records operations performed when the specified function is executed on the given tensors.
As a result, the resulting ScriptModule consistently executes the same traced graph for any input.
In conclusion, tracing can be a valid option in some cases, such as when there is no need to record any control-flow like if-statements or loops, but the scripting is preferred, and it is guaranteed to work in a more wide set of cases.
A call example of the compile Python API of Torch-MLIR is reported below.
\begin{lstlisting}[language=Python,label={lst:torch_mlir-compile}]
module = torch_mlir.compile(gnn_model, (x, features, adj),
                            output_type="linalg-on-tensors")
\end{lstlisting}

Once having obtained the compiled module, the expected behavior is to use one of the backends provided by torch-MLIR to make the inference.
This is not the flow followed in this thesis, because, as represented in the accelerator design flow, in Figure~\ref{fig:toolchain}, it is needed to export the Linalg representation for the next phase.
This can be done by simply saving the model to an MLIR file, as shown below.
\begin{lstlisting}[language=Python,label={lst:torch_mlir-export}]
with open("gnn_model.mlir", "w", encoding="utf-8") as outf:
    outf.write(str(module))
\end{lstlisting}

Only the GCN model implemented in PyTorch reached this phase of the toolchain.
During the research, much effort has been spent in trying to add support for the PyTorch Geometric framework to the toolchain.
Even if some innovation has been brought in this regard, there are still open points to work on.
For this reason, at the actual state, the proposed design flow only supports PyTorch as a high-level framework
It must be noted that no documentation existed and no previous works have ever used Torch-MLIR with GNN models.
All the examples and the work done by the Torch-MLIR community are related to Deep Neural Network models.

This thesis could represent the first work that explored the use of Torch-MLIR with GNN models.
Unfortunately, even if some innovation has been brought, such as the implementation of support of the constant of Tuple type, some more work is still required.
In particular, Torch-MLIR does not support the \lstinline{aten.scatter_add} operation, which, at the actual state, cannot be lowered to MLIR\@.
This operation is extensively used by PyTorch Geometric, leading to the incompatibility of the two elements.
The next step to add support for PyTorch Geometric to the proposed toolchain would be the implementation to Torch-MLIR of the lowering of the scatter add operation.

Another point discovered during this thesis is the fact that Torch-MLIR does not support the sparse tensor type. 
Each sparse tensor implemented in PyTorch, with the relative sparse operations, is lowered to MLIR to a dense tensor, losing all its representation's advantages.
A promising option to avoid this is Taco~\cite{taco} with its PyTaco APIs for Python.
The MLIR team started creating what is called MLIR-PyTACO~\cite{mlir-pytaco}, an end-to-end use case for the sparse tensor compiler, which can be used to lower sparse Tensor to MLIR\@.
It is important to clarify that, as introduced in~\cite{Bik_2022}, sparse tensors are supported by MLIR, with a dedicated sparse tensor dialect that uses intuitive annotations for different sparse tensor representations, such as CSR and COO.
What is not supported yet is the lowering through Torch-MLIR\@.
However, even if MLIR-PyTACO is still premature and in testing phase, interesting features could be brought by its advancement.

Even losing the sparse tensor representation, using the proper optimizations provided by the toolchain in combination with the higher computational performance of FPGAs, still make it possible to accelerate the GNN operations, as will be stated in the next Chapter.

\section{Synthesizer}
\label{sec:toolchain-synthesizer}%

The synthesizer represents the final step of the toolchain, which optimizes and synthesizes the MLIR representation, targeting FPGA\@.
This step includes SODA-OPT and PandA-Bambu, both introduced in Section~\ref{sec:soda}.
The following Subsections provide insight into what is happening internally to these two components.

\begin{figure}[t]
    \centering
    \subfloat[Compiler frontend\label{fig:soda-opt_flow}]{
        \includegraphics[height=0.5\textwidth]{Images/soda_opt_flow}
    }
    %\quad
    \hspace{0.03\textwidth}
    \subfloat[High-level synthesis backend\label{fig:bambu_flow}]{
        \captionsetup{width=.4\textwidth}
        \includegraphics[height=0.5\textwidth]{Images/bambu_flow}
    }
    \caption{Synthesizer: SODA-OPT and PandA-Bambu overview~\cite{9786533}}
    \label{fig:synthesizer_flow}
\end{figure}

\subsection{SODA-OPT}
\label{subsec:toolchain-soda_opt}%

SODA-OPT, as shown in Figure~\ref{fig:soda-opt_flow}, receives as input the MLIR representation of the model.
This step is primarily responsible for applying optimizations that can be exploited in the next step.
In particular, a subset of MLIR passes can be used to do so.
The output of SODA-OPT is an LLVM representation that serves as input to PandA-Bambu HLS\@.

Despite the remarkable capabilities of SODA-OPT, it should be noted that it does not support the entire set of dialects utilized within the MLIR ecosystem, such as the ml\_program or the Tensor dialect.
Consequently, an additional step is required wherein the representation obtained from Torch-MLIR is lowered to remove unsupported dialects.
In particular, once having successfully exported the MLIR representation using the \lstinline{torch_mlir.compile} API, the mlir-opt passes shown below have been used to remove the unsupported dialects by lowering them to supported ones, such as the memref.

\begin{lstlisting}[language=bash,label={lst:mlir-opt-remove}]
mlir-opt --canonicalize -convert-tensor-to-linalg \
         --empty-tensor-to-alloc-tensor \
         --eliminate-empty-tensors \
         -linalg-bufferize -arith-bufferize \
         -tensor-bufferize -func-bufferize \
         -finalizing-bufferize -buffer-deallocation \
         --buffer-results-to-out-params \
         --canonicalize -cse new_model.mlir
\end{lstlisting}

The next step consists in outlining the part of MLIR code to accelerate.
In general, the aim is to accelerate the whole function.
To do so, it is enough to modify the MLIR file by adding the \lstinline{soda.launch} and \lstinline{soda.terminator} flags at the beginning and end of the function, thus after the start of the forward and before the return statement.

SODA-OPT provides various passes that can be used to apply optimization to the outlined code.
In particular, it provides a subset of MLIR passes plus a set of passes tailored for soda optimizations.
The SODA-OPT passes are continuously evolving, trying to keep up with rapid advancement and innovation within the domain of MLIR\@.

An essential part of this research has been the analysis of the GNN model, the understanding of its bottlenecks, and the consequent identification of the optimization having the biggest impact on performance, without increasing too much the area of the accelerator.

The first part of this analysis has been conducted on PyTorch.
Profiling the inference function of the GCN model, the result showed that, in general, nearly 60\% of the time required to make a prediction was used for the matrix multiplication operation.
For this reason, an important part of this thesis is represented by research on how to accelerate matrix multiplication using SODA-OPT passes to then accelerate the GCN inference.

This analysis has been conducted using a subset of the Cora dataset, and the first pass used transforms the linalg to affine dialect, resulting in three nested affine loops shown below.

\begin{lstlisting}[label={lst:affine-mul}, caption=Matrix multiplication in MLIR affine dialect]
affine.for %arg3 = 0 to 15 {
affine.for %arg4 = 0 to 16 {
  affine.for %arg5 = 0 to 15 {
    %0 = affine.load %arg0[%arg3, %arg5] : memref<15x15xf32>
    %1 = affine.load %arg1[%arg5, %arg4] : memref<15x16xf32>
    %2 = affine.load %arg2[%arg3, %arg4] : memref<15x16xf32>
    %3 = arith.mulf %0, %1 : f32
    %4 = arith.addf %2, %3 : f32
    affine.store %4, %arg2[%arg3, %arg4] : memref<15x16xf32>
  }
}
}
\end{lstlisting}

The result of this research showed that the most impressive improvement in terms of performance is given by the loop unrolling technique.
This optimization perfectly allows to exploit the extreme parallelism available on FPGAs.
The right choice is not to continuously unroll until having no more loops in the code.
The solution is to pick the right trade-off between performance reduction and the area of the matrix multiplication accelerator.

SODA-OPT, after having identified the key code regions, having outlined them into separate MLIR modules, and having applied the transformation passes to the MLIR input, optimizes the portion of code selected for hardware acceleration with a progressive lowering through different MLIR dialects.
As a final result, the input code is translated into an LLVM intermediate representation intentionally restructured for hardware synthesis.

\subsection{PandA-Bambu}
\label{subsec:toolchain-panda_bambu}%

PandA-Bambu represents the last phase of the synthesis.
As represented in Figure~\ref{fig:bambu_flow}, it receives the LLVM representation as input, and after having applied some optional low-level optimizations, it performs the typical steps of HLS introduced in Section~\ref{sec:hls}.

This stage represented another important part of this research.
PandA-Bambu allows the specification of different optimizations and settings that can have a big impact on accelerator performance.
Some optimization techniques have been explored, but the most evaluated setting is the number of memory channels.

In particular, a comparative analysis has been performed between different number of memory channels, including the minimum, 2 channels, and the maximum, 32 channels.
The latter option uses an external memory for the accelerator, which allows to better exploit the high level of parallelism that can be achieved using the loop unrolling technique, but at the same time, more cycles and area for loading data are required.
It is the case of a trade-off between the number of channels and the required number of data load cycles.
The conducted study revealed that there is a point, represented by a specific number of parallel operations, after which using an external memory with 32 channels becomes beneficial.

PandA-Bambu also offers the possibility to apply low-level loop unrolling, but this option has been disabled to be able to evaluate in isolation the performance impact of the SODA-OPT loop unrolling technique.
Bambu provides also the possibility to export some graphs that have been used for this research.
One important file is represented by the HLS graph, which shows the computation states and transitions, the number of cycles needed by each operation to complete, and other information that has been really useful to study and understand the impact of both SODA-OPT and Bambu optimization settings.

The LLVM intermediate representation taken as input from PandA-Bambu is received by the Clang compiler frontend which builds an internal IR to perform the HLS steps.
After having applied the specified optimizations, the generated design in an HDL is given as output.
As a result, after traversing through each stage of the proposed toolchain's process, the accelerator, i.e., the Bambu's output, represents the final output, tailored to target and maximize performance on cutting-edge FPGA architectures.

\section{Discussion}
\label{sec:toolchain-discussion}%

This Chapter presented the most important contribution of this thesis, an FPGA toolchain to create GNN hardware accelerators starting directly from PyTorch high-level framework, with the possibility to use different optimizations to improve performance depending on application bottlenecks.

The most crucial advantage of the proposed toolchain is that it allows obtaining an accelerator without any knowledge of hardware design and implementation.

During this research, the first accelerator developed was tailored for the matrix multiplication operation.
The second accelerator, which exploited the result of the first one, improved the inference performance of the GCN model.

In conclusion, the presented toolchain allows for the easy development of the GNN hardware accelerator.
No hardware design knowledge is required, and it can be used to develop accelerators for any application of all GNN models.



% ##########################################################################
% CHAPTER SIX - EXPERIMENTAL RESULTS
% ##########################################################################


    \chapter{Experimental Procedures and Results}
    \label{ch:chapter_six}%
    This chapter presents all the experiments performed about matrix multiplication acceleration and GNN acceleration along with the achieved results.

All the CPU experiments have been conducted using an Intel Core i9, with 8 cores and a frequency of 2,3 GHz.
On the other hand, the synthesis experiments targeted an AMD Virtex UltraScale+ (Alveo U280) FPGA\@.

\section{Model analysis and profiling}
\label{sec:model-analysis}%

\begin{table}[b]
\centering
    \begin{tabular}{|p{6em} c c c c|}
    \hline
%    \rowcolor{black!40}
    \textbf{Name} & \textbf{Self CPU \%} & \textbf{Self CPU} & \textbf{CPU total \%} & \textbf{CPU total} \T\B \\
    \hline \hline
    \textbf{aten::mm} & 50,25\% & 1,012$ms$ & 89,72\% & 1,807$ms$ \T\B\\
    \hline
    \textbf{aten::addmm} & 36,30\% & 731,0$\mu s$ & 37,04\% & 746,0$\mu s$ \T\B\\
    \hline
    \textbf{aten::add} & 4,67\% & 94,0$\mu s$ & 4,67\% & 94,0$\mu s$ \T\B\\
    \hline
    \multicolumn{5}{|c|}{\makecell{\lstinline{aten::mm} matrix multiplication; \lstinline{aten::add} matrix sum; \lstinline{aten::addmm} \\matrix multiplication plus matrix sum.}} \T\B\\
    \hline
    %\textbf{aten::\_log\_softmax} & 2,98\% & 60,0us & 2,98\% & 60,0us \T\B\\
    %\hline
    \end{tabular}
    \\[10pt]
    \caption{Excerpt of GCN model inference profiling result}
    \label{tab:gcn_profiling}
\end{table}

As already anticipated in Section~\ref{sec:toolchain-pytorch}, the GCN model is implemented in PyTorch, and it is characterized by two convolutional layers, a ReLU and a dropout functions.
The forward function of each layer contains two matrix multiplications, and one of the two is a sparse multiplication.

The first step to understand how to accelerate the PyTorch GCN model was to analyze and profile it.
Table~\ref{tab:gcn_profiling} shows the results extracted from a run of the PyTorch profiler, which measures the ATen operations used at a lower level of PyTorch.
In particular, ATen~\cite{aten} is a tensor library that serves as the foundation for most Python and C++ interfaces within PyTorch.
It provides a central Tensor class that encompasses a multitude of operations, with the Tensor class dynamically selecting the appropriate one based on its type.

Among the profiled operations, \lstinline{aten::mm} performs a multiplication between two tensors, \lstinline{aten::addmm} performs a multiplication between two tensors and then another tensor is added to the result and finally \lstinline{aten::add} returns the sum of two tensors provided as input.
The distinction between \textit{self CPU time} and \textit{total CPU time} lies in the fact that self CPU time does not contain the time spent in child operator calls, whereas total CPU time contains it, considering that operators can invoke other operators.
It is clear that the bottleneck and the most time-consuming operation is the matrix multiplication.
In particular, more than 50\% of the self CPU time is used by matrix multiplication, while, considering the child operator calls, this percentage represents nearly the 90\%.
This result clearly justifies a special focus on matrix multiplication acceleration.

\section{Matrix multiplication acceleration}
\label{sec:matmul-acceleration}%

Matrix multiplication, introduced in Section~\ref{sec:matmul-synthesis-heiristic}, is a well-known algorithm.
It consists of multiplying two compatible matrices to obtain the result matrix.
A lot of work has been done to try to improve its performance on different architectures~\cite{DBLP:journals/corr/abs-2003-00532, opt_cuda_matmul}.
The following subsections explains how this thesis improved matmul performance exploiting the proposed toolchain.

\subsection{PyTorch matrix multiplication benchmark}
\label{subsec:pytorch-matmul-bench}%

PyTorch provides different matrix representations and different matrix multiplication functions.
The one considered in this Subsection are \textit{torch.mm} and \textit{torch.spmm}.
The former function multiplies two dense matrices, but it also supports COO and CSR representation.
The latter, instead, is typically used for sparse matrix multiplications, in which one of the two matrices, or both, are saved using sparse representations.

Table~\ref{tab:torch-mm-benchmark} represents a benchmark for the dense matrix multiplication between a first matrix of size $15 \times 15$ and a second matrix of size $15 \times 16$, both composed by float32 elements.
The chosen sizes for input matrices are not arbitrary; they have been specifically selected due to their extensive use in the following experiments, requiring an accurate benchmark for comparison purposes.
In particular, the table shows five different runs, each composed by a different number of executions.
Then, the final time is computed as the average of the five different runs.
Additionally, this average has been computed five times using increasing number of executions.
The results shows that the variance of the runs of the last experiments is lower than the others, fixing to ten millions a sufficient number of executions for good accuracy and stability.

As expected, given the globally high amount of executions, the five average execution times are similar between them, and the variance decreases as the number of executions increases.
In conclusion, the time needed by a dense matrix multiplication between two matrices of the given size can be considered equal to 1.608$\mu s$.
%The Python timing corresponding to the accelerated matrix multiplication for each experiment has been computed using an average of ten million executions.

Since the GCN model uses both dense and sparse matrix multiplication functions, Table~\ref{tab:torch-matmul-comparison} shows the times needed by both functions according to different representations of the two input matrices A and B\@.
Different experiments have been performed using randomly generated input matrices; they are composed by float32 elements and different input sized and levels of sparsity have been analyzed.

\begin{table}[t]
\centering
    \resizebox{\textwidth}{!}{
    \begin{tabular}{|p{6em} c c c c c c|}
    \hline
    \thead{Executions} & \thead{Run1} & \thead{Run2} & \thead{Run3} & \thead{Run4} & \thead{Run5} & \thead{\makecell{Avg. time}} \T\B \\
    \hline \hline
    \makecell{2E06} & 1.587E-06 & 1.568E-06 & 1.567E-06 & 1.594E-06 & 1.572E-06 & 1.578E-06 \T\B\\
    \hline
    \makecell{4E06} & 1.598E-06 & 1.585E-06 & 1.592E-06 & 1.602E-06 & 1.599E-06 & 1.595E-06 \T\B\\
    \hline
    \makecell{6E06} & 1.601E-06 & 1.614E-06 & 1.603E-06 & 1.603E-06 & 1.608E-06 & 1.606E-06 \T\B\\
    \hline
    \makecell{8E06} & 1.608E-06 & 1.607E-06 & 1.600E-06 & 1.614E-06 & 1.617E-06 & 1.609E-06 \T\B\\
    \hline
    \makecell{10E06} & 1.614E-06 & 1.603E-06 & 1.603E-06 & 1.613E-06 & 1.607E-06 & 1.608E-06 \T\B\\
    \hline
    \end{tabular}
    }
    \\[10pt]
    \caption[Benchmark of \textit{torch.mm} PyTorch function]{Benchmark of \textit{torch.mm} PyTorch function. Five runs using different number of executions and their average, unit of measure is second.}
    \label{tab:torch-mm-benchmark}
\end{table}

\begin{table}[t]
\centering
    \resizebox{\textwidth}{!}{
    \begin{tabular}{|p{6em} c c c c c c c c|}
    \hline
    \textbf{Function} & \textbf{Sizes} & \textbf{Sparsity} & \textbf{Runs} & \textbf{Dense$\times$Dense} & \textbf{COO$\times$Dense} & \textbf{COO$\times$COO} & \textbf{CSR$\times$Dense} & \textbf{CSR$\times$CSR} \T\B \\
    \hline \hline
    \textbf{torch.mm} & 15$\times$15, 15$\times$16 & 0.9 & 10E06 & \textbf{1.599E-06} & 2.844E-06 & 1.658E-05 & 1.058E-05 & 8.632E-05 \T\B\\
    \hline
    \textbf{torch.spmm} & 15$\times$15, 15$\times$16 & 0.9 & 10E06 & 1.934E-06 & 3.385E-06 & 1.750E-05 & 1.183E-05 & 1.746E-05 \T\B\\
    \hline \hline
    \textbf{torch.mm} & 150$\times$150, 150$\times$16 & 0.9 & 1E06 & \textbf{5.220E-06} & 4.303E-05 & 1.058E-04 & 1.384E-05 & 1.393E-04 \T\B\\
    \hline
    \textbf{torch.spmm} & 150$\times$150, 150$\times$16 & 0.9 & 1E06 & 6.593E-06 & 4.659E-05 & 1.143E-04 & 1.461E-05 & 1.090E-04 \T\B\\
    \hline \hline
    \textbf{torch.mm} & 150$\times$150, 150$\times$16 & 0.99 & 1E06 & 5.752E-06 & 7.039E-06 & 1.887E-05 & 1.256E-05 & 8.730E-05 \T\B\\
    \hline
    \textbf{torch.spmm} & 150$\times$150, 150$\times$16 & 0.99 & 1E06 & \textbf{4.678E-06} & 7.797E-06 & 1.883E-05 & 1.214E-05 & 1.847E-05 \T\B\\
    \hline \hline
    \textbf{torch.mm} & 2708$\times$2708, 2708$\times$16 & 0.999 & 100E03 & 1.288E-03 & 2.030E-04 & 1.325E-04 & \textbf{3.377E-05} & 1.073E-04 \T\B\\
    \hline
    \textbf{torch.spmm} & 2708$\times$2708, 2708$\times$16 & 0.999 & 100E03 & 1.255E-03  & 2.012E-04 & 1.316E-04 & 3.661E-05 & 1.186E-04 \T\B\\
    \hline
    \end{tabular}}
    \\[10pt]
    \caption[Comparison between dense and sparse PyTorch matmul functions]{Comparison between dense and sparse PyTorch matmul functions, computed as average of multiple runs. Times unit of measure is seconds.}
    \label{tab:torch-matmul-comparison}
\end{table}

%\begin{figure}[t]
%    \centering
%    \includegraphics[height=0.4\textwidth]{Images/torch-mm_benchmark}
%    \caption{Benchmark of \textit{torch.mm} PyTorch function}
%    \label{fig:torch-mm_benchmark}
%\end{figure}

It is clear that sparse matrix multiplication, both with dense and sparse matrix representations, does not increase performance on CPU architecture when the size of input matrices is small.
The disadvantage of using sparse matrix multiplication on CPU becomes more evident as the size of the input matrices increases but the level of sparsity remains constant.
However, when the size of input matrices increases considerably and the level of sparsity is very high, using sparse representations, in particular CSR, can bring a significant advantage.

Even if the following experiments do not take advantage of sparse matrix computations, the utilization of custom optimizations for the analyzed GCN has enabled the acceleration of inference, balancing the absence of sparse matrix computations.

\subsection{FPGA-accelerated matrix multiplication}
\label{subsec:optimization-comparison}%

Before applying any optimization, it is necessary to understand the difference between PyTorch matrix multiplication operation and the baseline accelerator.
To obtain the matmul accelerator, the process began with following the initial steps of the toolchain using the GCN model.
After having made the model compatible with TorchScript, its linalg representation has been obtained through Torch-MLIR compilation.
Subsequently, only the matmul code of the generated MLIR representation has been outlined through the use of SODA annotation.
Following this, the LLVM-generated file has been synthesized with PandA-Bambu using baseline attributes, including the \lstinline{-fno-unroll-loops} flag, two memory channels, and the \lstinline{ALL_BRAM} option.
Figure~\ref{fig:pytorch-accelerator-comparison} shows the result of this analysis, which reveal the fact that the cycle counts required by the accelerator for each experiment align perfectly with the anticipated expectations outlined in Section~\ref{sec:matmul-synthesis-heiristic}.
Specifically, dividing the cycle count of the first experiment of the table and the final one by the total count of loop iterations effectively confirms a consistent result of approximately 7 in both instances, confirming that the number of cycles per each loop iteration stayed constant, as shown by the second part of Equation~\ref{eq:number-cycles}.
However, this constant behavior limits the utilization of the possibilities presented by FPGA technology, which encompass substantial parallelization potential and concurrent memory accesses.

The PyTorch times have been recorded by averaging five measurements each of ten millions executions.
The baseline accelerator is much faster than PyTorch when matrices are relatively small.
The reason behind this behavior is that the generated accelerator simplifies data loading by assuming that all data is available in BRAM. By default, the system uses two channels and stores all memory objects in BRAMs.
However, an alternative approach, explored in this thesis, involves employing a greater number of memory channels while utilizing external memory.
The advantage of the accelerator in terms of performance decreases as the size of the input matrices increase, until reaching a point in which the accelerator becomes slower than PyTorch solution.

%\begin{table}[t]
%\centering
%    \begin{tabular}{|p{9em} c c c c  |}
%    \hline
%    \textbf{Input sizes} & \textbf{Torch.mm (s)} & \textbf{Runtime (s)} & \textbf{Cycles} & \textbf{SpeedUp} \T\B \\
%    \hline \hline
%    \textbf{15$\times$15, 15$\times$16} & 1.60E-06  & 96.49E-09 & 25,697 & 16.58 \T\B\\
%    \hline
%    \textbf{30$\times$30, 30$\times$16} & 1.73E-06  & 351.18E-09 & 101,792 & 4.92 \T\B\\
%    \hline
%    \textbf{60$\times$60, 60$\times$16} & 2.48E-06  & 1.46E-06 & 405,182 & 1.69 \T\B\\
%    \hline
%    \textbf{90$\times$90, 90$\times$16} & 4.55E-06  & 3.15E-06 & 910,172 & 1.44 \T\B\\
%    \hline
%    \textbf{120$\times$120, 120$\times$16} & 4.79E-06  & 5.62E-06 & 1,616,762 & 0.85 \T\B\\
%    \hline
%    \textbf{150$\times$150, 150$\times$16} & 5.16E-06  & 9.07E-06 & 2,524,952 & 0.56 \T\B\\
%    \hline
%    \end{tabular}
%    \\[10pt]
%    \caption{Comparison between dense and sparse PyTorch matmul functions}
%    \label{tab:pytorch-accelerator-comparison}
%\end{table}

\begin{figure}[t]
    \centering
    \includegraphics[height=0.4\textwidth]{Images/matmul_comparison}
    \caption{Performance comparison between PyTorch matmul function and accelerator}
    \label{fig:pytorch-accelerator-comparison}
\end{figure}

This behaviour can be attributed to the fact that PyTorch times have been recorded using all the eight available threads on the machine.
So, it is expected that the \lstinline{torch.mm} function exploits more parallelism with respect to the accelerator, and this advantage is more evident when the potential level of parallelism increases.
For this reason, the optimizations discussed in Section~\ref{subsec:toolchain-soda_opt} and in Section~\ref{subsec:toolchain-panda_bambu} need to be applied to exploit more parallelism.

SODA-OPT, as introduced in Subsection~\ref{subsec:toolchain-soda_opt}, offers the possibility to make different types of unrolling, among which the full unroll which completely unroll the innermost loop, and the partial unroll up to an arbitrary factor.
Moreover, PandA-Bambu introduces several optimization options, one of which, as already anticipated above, involves expanding the number of memory channels in use by exploiting an external memory instead of a BRAM.

The combination of loop unrolling and multiple memory channels offers both advantages and disadvantages, resulting in a trade-off.
Loop unrolling enhances parallelization, thereby reducing computational time.
However, especially when utilizing more than two memory channels, it increases the number of parallel processing elements, leading to a larger area footprint.
Additionally, external memory allows for up to 32 memory channels, enabling the simultaneous loading of 32 variables.
Nonetheless, accessing data from external memory requires more load cycles compared to accessing internal memory.
In particular, with BRAM both load and store requires one cycle, instead, loads with external memory still require one cycle, but stores require 2 cycles to complete.

\begin{figure}[t]
    \centering
    \subfloat[Input matrices $15\times15$, $15\times16$\label{fig:matmul-optimization-comparison15}]{
        \includegraphics[height=0.4\textwidth]{Images/matmul_comparison15}
    }
    %\quad
    \hspace{0.15\textwidth}
    \subfloat[Input matrices $30\times30$, $30\times16$\label{fig:matmul-optimization-comparison30}]{
        \includegraphics[height=0.4\textwidth]{Images/matmul_comparison30}
    }
    \caption{Matrix multiplication optimization comparison}
    \label{fig:matmul-optimization-comparison}
\end{figure}

Figure~\ref{fig:matmul-optimization-comparison} show the performance of two matrix multiplication accelerators operating on different sizes of the input matrices.
Figure~\ref{fig:matmul-optimization-comparison15} show results of a multiplication between two matrices of size $15\times15$ and $15\times16$, while Figure~\ref{fig:matmul-optimization-comparison30} shows results of a multiplication between two matrices of size $30\times30$ and $30\times15$.
In both figures, on the x-axis there is the unrolling factor used and on the y-axis the resulting number of cycles.
The baseline computation does not use unrolling, instead when there are two factors it means that the first one can be considered as a full unroll of the innermost cycle, while the second one is the unrolling factor applied to the original second innermost cycle.

In both cases, when the parallelization is not high, thus the unrolling factor is small, using two channels and storing all memory objects in BRAMs is the best option since a minor number of cycles is required to perform the computation.
Using 16 channels seems to be similar to use 32 channels, but when parallelization is high the latter option uses fewer cycles.

The best trade-off is achieved using an unrolling factor for which the number of cycles needed by the accelerator with 32 memory channels is lower than the number of cycles needed by the accelerator with 2 memory channels.
In the first case, in Figure~\ref{fig:matmul-optimization-comparison15}, this objective is achieved with two unrolling factors of 15 and 8, while in the second case is achieved with unrolling factors of 30 and 8.

These results can be used to set a heuristic useful to identify the number of parallel unrolled loop iterations that makes the use of 32 channels with external memory preferable to the use of 2 channels with BRAM:

\begin{equation}
    \label{eq:factor-relation}
        i \cdot j \cdot k \geq 2 \cdot \sqrt {M \cdot N \cdot R}
\end{equation}

where $M$, $N$ and $R$ are the sizes of the three nested loops, as defined in Algorithm~\ref{alg:var}, and $i$, $j$ and $k$ are their respective loop unrolling factors, following the rule $j \neq 1 \iff i=M \land k \neq 1 \iff j=N$.

Equation~\ref{eq:factor-relation} captures the relations of the two results.
In fact, using 15 and 8 as unrolling factors means having $15 \cdot 8 = 120$ parallel loop iterations.
Instead, using 30 and 8 as unrolling factors means having $30 \cdot 8 = 240$ parallel loop iterations.
Additionally, the total amount of possible parallel loop iterations in a matrix multiplication between two matrices of sizes $15\times15$ and $15\times16$ is equal to $15 \cdot 16 \cdot 15 = 3,600$.
Meanwhile, the total amount of possible parallel loop iterations in a matrix multiplication between two matrices of sizes $30\times30$ and $30\times16$ is $30 \cdot 16 \cdot 30 = 14,400$.
Even if the two number of parallel loop iterations representing the changing point of the trade-off, 120 and 240, are different, they are related by Equation~\ref{eq:factor-relation}.

Equation~\ref{eq:factor-relation} should not be taken as an infallible rule, it is the outcome of this experimental phase and should be used as a discriminant to decide when to use thirty-two memory channels instead of two.
In conclusion, the generalized rule, outcome of this comparative analysis, conducted using different sizes of input matrices, is that the number of parallel unrolled loop iterations that justify the use of 32 channels is given by Equation~\ref{eq:factor-relation}.

\subsection{GCN accelerator evaluation}
\label{subsec:gcn_accelerator_evaluation}%

\begin{table}[t]
\centering
    \begin{tabular}{|p{4em} c c c c c|}
    \hline
    \textbf{Name} & \textbf{Nodes} & \textbf{Words} & \textbf{Links} & \textbf{Task} & \textbf{Classes} \T\B \\
    \hline \hline
    \textbf{Cora} & 2708  & 1433 & 5429 & Multiclass classification & 7 \T\B\\
    \hline
    \textbf{Cora15} & 15  & 15 & 3 & Multiclass classification & 7 \T\B\\
    \hline
    \textbf{Cora30} & 30  & 30 & 4 & Multiclass classification & 7 \T\B\\
    \hline
    \textbf{Cora60} & 60  & 60 & 8 & Multiclass classification & 7 \T\B\\
    \hline
    \textbf{Cora90} & 90  & 90 & 18 & Multiclass classification & 7 \T\B\\
    \hline
    \textbf{Cora120} & 120  & 120 & 22 & Multiclass classification & 7 \T\B\\
    \hline
    \textbf{Cora150} & 150  & 150 & 37 & Multiclass classification & 7 \T\B\\
    \hline
    \end{tabular}
    \\[10pt]
    \caption{Cora sub-dataset used for GCN infernce}
    \label{tab:dataset-definition}
\end{table}

In this Subsection, the toolchain and the proposed optimizations are applied to the GCN model.

%\begin{table}[t]
%\centering
%    \begin{tabular}{|p{4em} c c c c|}
%    \hline
%    \thead{Dataset} & \thead{CPU PyTorch(s)} & \thead{Optimizations} & \thead{Runtime(s)} & \thead{SpeedUp} \T\B \\
%    \hline \hline
%    \makecell{Cora15} & 59.25E-06 & \makecell{Full Unroll} & 523.24E-09 & 113.23 \T\B\\
%    \hline
%    \makecell{Cora30} & 66.42E-06 & \makecell{Full Unroll} & 1.79E-06 & 37.10 \T\B\\
%    \hline
%    \makecell{Cora60} & 69.75E-06 & \makecell{Full Unroll} & 7.13E-06 & 9.78 \T\B\\
%    \hline
%    \makecell{Cora90} & 88.88E-06 & \makecell{Full Unroll} & 14.90E-06 & 5.96 \T\B\\
%    \hline
%    \makecell{Cora120} & 98.32E-06 & \makecell{Full Unroll} & 29.64E-06 & 3.31 \T\B\\
%    \hline
%    \makecell{Cora150} & 115.03E-06 & \makecell{Full Unroll} & 41.12E-06 & 2.79 \T\B\\
%    \hline
%    \end{tabular}
%    \\[10pt]
%    \caption{GCN inference time comparison}
%    \label{tab:GCN-inference-pytorch-accelerator-comparison}
%\end{table}

\begin{figure}[t!]
    \centering
    \includegraphics[height=0.4\textwidth]{Images/gcn_forward_comparison}
    \caption{GCN inference PyTorch-Accelerator comparison}
    \label{fig:gcn-inference-comparison}
\end{figure}

Figure~\ref{fig:gcn-inference-comparison} shows the result of a comparative analysis between the PyTorch CPU time and the optimized FPGA accelerator time to perform GCN inference.
All the experiments have been performed using a subset of the Cora dataset introduced in Subsection~\ref{subsec:toolchain-inputs}, as detailed in Table~\ref{tab:dataset-definition}.
The PyTorch times have been acquired averaging one million of execution time measurements using the PyTorch built-in benchmark API\@.

The accelerator times are evaluated using both baseline and optimized settings, and Figure~\ref{fig:gcn-inference-cycles-comparison} and Table~\ref{tab:GCN-inference-accelerators-comparison} show how the accelerator has been affected by the unrolling technique with respect to the baseline performance.
The optimized settings uses two channels and one full unrolling of the innermost loop, and it obviously requires more area than the baseline accelerator, but it is computationally faster; being matrix multiplication the most time-consuming operation of the GCN, thanks to loop unrolling technique, the speedup increases as the size of the dataset, and thus the number of nodes in the graph, increases.
The used optimized setting has been preferred to the alternative considered setting with thirty-two channels and two full loop unrolling, because it achieves high speedup without increasing too much area requirements.

\begin{table}[t]
\centering
    \resizebox{\textwidth}{!}{
    \begin{tabular}{|p{4em} c c c c c c c c c|}
    \hline
    \thead{Dataset} & \thead{Optimizations} & \thead{Cycles} & \thead{Slices} & \thead{Luts} & \thead{Registers} & \thead{DSPs} & \thead{BRAMs} & \thead{Frequency(MHz)} & \thead{SpeedUp} \T\B \\
    \hline \hline
    \makecell{Cora15} & \makecell{Baseline} & 115,852 & 8,307 & 32,927 & 32,195 & 16 & 256 & 147.77 & - \T\B\\
    \hline
    \makecell{Cora15} & \makecell{Full Unroll} & 93,705 & 8,338 & 36,265 & 35,164 & 44 & 256 & 179.08 & 1.23 \T\B\\
    \hline \hline
    \makecell{Cora30} & \makecell{Baseline} & 385,874 & 7,457 & 30,379 & 26,925 & 16 & 256 & 158.25 & - \T\B\\
    \hline
    \makecell{Cora30} & \makecell{Full Unroll} & 301,800 & 12,448 & 50,907 & 50,558 & 74 & 256 & 168.12 & 1.27\T\B\\
    \hline \hline
    \makecell{Cora60} & \makecell{Baseline} & 1,402,860 & 6,928 & 30,115 & 24,287 & 16 & 256 & 158.07 & - \T\B\\
    \hline
    \makecell{Cora60} & \makecell{Full Unroll} & 1,064,580 & 21,726 & 94,025 & 77,956 & 134 & 256 & 149.20 & 1.31 \T\B\\
    \hline \hline
    \makecell{Cora90} & \makecell{Baseline} & 3,051,630 & 6,769 & 29,899 & 25,966 & 16 & 256 & 166.69 & - \T\B\\
    \hline
    \makecell{Cora90} & \makecell{Full Unroll} & 2,298,510 & 32,046 & 154,917 & 108,770 & 194 & 256 & 154.20 & 1.32 \T\B\\
    \hline \hline
    \makecell{Cora120} & \makecell{Baseline} & 5,332,200 & 8,045 & 30,441 & 25,878 & 16 & 256 & 145.79 & - \T\B\\
    \hline
    \makecell{Cora120} & \makecell{Full Unroll} & 3,987,840 & 43,187 & 218,178 & 135,961 & 254 & 256 & 134.49 & 1.33 \T\B\\
    \hline \hline
    \makecell{Cora150} & \makecell{Baseline} & 8,244,570 & 7,499 & 29,390 & 26,015 & 16 & 264 & 163.61 & - \T\B\\
    \hline
    \makecell{Cora150} & \makecell{Full Unroll} & 6,136,470 & 27,002 & 115,463 & 143,916 & 26 & 264 & 149.20 & 1.34 \T\B\\
    \hline
    \end{tabular}}
    \\[10pt]
    \caption{GCN inference time comparison}
    \label{tab:GCN-inference-accelerators-comparison}
\end{table}

\begin{figure}[t!]
    \centering
    \includegraphics[height=0.4\textwidth]{Images/gcn_forward_cycles_comparison}
    \caption{GCN inference number of cycles comparison}
    \label{fig:gcn-inference-cycles-comparison}
\end{figure}

The results of this final evaluation are incredibly encouraging.
It is important to mention that the optimized accelerator uses two channels with on-chip BRAMs.
Only the computational time of the accelerator is evaluated and results are based on the assumption that all data are loaded from BRAMs, requiring two cycles to read data and one cycle to store data.
The speedup of the accelerator is still being affected by the size of the input matrices, but this was an expected result since the unrolling factor used is contained.
The accelerator's computational time is significantly lower with respect to the one measured on CPU with PyTorch, and for bigger dataset sizes the technique presented in Subsection~\ref{subsec:optimization-comparison} can be used, accelerating even more the computational performance of the accelerator.

The area of the accelerator, instead, is obviously affected by the sizes of the input matrices and of the unrolling factor.
This because more the matrices are big, more the parallel loop iterations will be and thus the parallel processing elements.

However, the possibilities offered by the proposed toolchain are various.
It is possible to use less memory channels with a lower loop unrolling factor to still have a big positive impact on performance, but containing the accelerator area requirements.
The Equation~\ref{eq:factor-relation} can be adapted also to this specific case, by considering the number of load and store and the size of the different loop iterations, becoming a valid indicator to decide which optimization to apply according to the dataset size.





% ##########################################################################
% CHAPTER SEVEN - CONCLUSION
% ##########################################################################

    \chapter{Conclusion and Future Developments}
    \label{ch:conclusions}%
    This thesis tackled the challenge of accelerating Graph Neural Network inference by leveraging High-Level Synthesis techniques targeting FPGAs.

In Chapter~\ref{ch:chapter_five}, the design of the proposed toolchain was introduced, which allows to obtain a GNN inference accelerator starting directly from PyTorch.
PyTorch stands as one of the foremost high-level frameworks for neural network implementation, extensively recognized and employed within the community, making the proposed toolchain suitable for different applications.

The results of this research significantly contribute to the field of GNN acceleration, introducing a new perspective about how it is possible to obtain hardware accelerators for GNNs even without having any hardware design knowledge.
The toolchain offers different possibilities, and provides various optimization passes that can be used in the synthesizer step to fine-tune the accelerator capabilities.

Within the spectrum of optimizations made available by this toolchain, this thesis primarily delved into two key techniques:: the loop unrolling technique of SODA-OPT and the parallel memory access of PandA-Bambu with an external memory of thirty-two channels.
These optimizations allowed to achieve encouraging and promising result in accelerating the inference of the GCN model analysed.
By studying and understanding the model bottlenecks, it is possible to achieve consistent improvements.

Furthermore, the applicability of this toolchain extends to the industry area, making it a readily accessible resource for companies looking to explore and capitalize on the domain of GNN acceleration.

An equally noteworthy contribution of this thesis lies in the innovation brought to SODA-OPT and PandA-Bambu.
This research represents the first work exploring the loop-unrolling technique in combination with the recent added feature of using more than two memory channels, providing consistent result and insights for future studies.

Lastly, this study has also made substantial contribution in enhancing Torch-MLIR.
A new feature, the support of the constant of Tuple type, have been added and different areas of improvement have been identified, where more work would be needed to implement functionalities for the complete support of PyTorch Geometric.
Before this research, no examples were available on how to use Torch-MLIR with Graph Neural Networks, and the compatibility of PyTorch Geometric and Torch-MLIR was still an unexplored area.

\section{Future developments}
\label{sec:future-dev}%



%##########################################################################
%	BIBLIOGRAPHY
%##########################################################################

    \addtocontents{toc}{\vspace{2em}} % Add a gap in the Contents, for aesthetics
    \bibliography{Thesis_bibliography} % The references information are stored in the file named "Thesis_bibliography.bib"

%-------------------------------------------------------------------------
%	APPENDICES
%-------------------------------------------------------------------------

    \cleardoublepage
    \addtocontents{toc}{\vspace{2em}} % Add a gap in the Contents, for aesthetics
    \appendix

%
%    \chapter{Appendix A}
%    If you need to include an appendix to support the research in your thesis, you can place it at the end of the manuscript.
%    An appendix contains supplementary material (figures, tables, data, codes, mathematical proofs, surveys, \dots)
%    which supplement the main results contained in the previous chapters.
%
%
%    \chapter{Appendix B}
%    It may be necessary to include another appendix to better organize the presentation of supplementary material.

% LIST OF FIGURES
    \listoffigures

% LIST OF TABLES
    \listoftables

% LIST OF ALGORITHMG
    \listofalgorithms

% LIST OF LISTINGS
    \lstlistoflistings



% LIST OF SYMBOLS
% Write out the List of Symbols in this page
    \chapter*{List of Symbols} % You have to include a chapter for your list of symbols (
    \begin{table}[H]
        \centering
        \begin{tabular}{lll}
            \textbf{Notation} & \textbf{Description} \\\hline\\[-9px]
            $\mathcal{G} = (V, E)$        & The input graph for the GNN  \\[2px]
            $V$ & Set of vertices of the graph \\[2px]
            $E$ & Set of edges of the graph \\[2px]
            $\mathcal{N}(v)$ & Set of neighbors of vertex $v$ \\[2px]
            $A \in \mathbb{R}^{N \times N}$        & Adjacency matrix of $\mathcal{G}$ ($N$ : number of nodes)  \\[2px]
            $\tilde{D}$ & Degree matrix of the graph \\[2px]
            $W^{(l)}$ & Weight matrix of the neural network ($l$ : layer) \\[2px]
            $H^{(l)}$ & Input node features matrix ($l$ : layer) \\[2px]
            $h_v$ & Node representation of node $v$ \\[2px]
            $\epsilon$ & Learnable parameter or fixed scalar \\[2px]
            $I$ & Identity matrix \\[2px]

        \end{tabular}\label{tab:symbols_table}
    \end{table}

% ACKNOWLEDGEMENTS
    \chapter*{Acknowledgements}
    Acknowledgements here...

    \cleardoublepage

\end{document}
